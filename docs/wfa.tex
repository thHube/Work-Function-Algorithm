\documentclass[a4paper, 11pt]{article}

\usepackage[utf8]{inputenc}
\usepackage[italian]{babel}

\usepackage{amsmath}
\usepackage{amsthm}

\begin{document}

\title{Una semplice prova della $(2k - 1)$-competitività di $\omega$-WFA}
\maketitle

Vogliamo mostrare come l'ottimizzazione dell'algoritmo \textsc{Work} con una 
finestra di ricerca sia $(2k - 1)$-competitivo esattamente come l'algoritmo 
originale modulo alcune assunzioni. In primo luogo considereremo uno spazio
metrico finito, la cui distanza massima è $\Delta$ inoltre consideriamo una
discretizzazione di questo spazio tale che la distanza minima tra le richieste
si $\delta$. In generale queste assunzioni non sono limitanti dal momento che 
le implementazioni effettive sul calcolatore hanno intrinsecamente queste 
limitazioni.

La prima scelta che operiamo è la dimensione della finestra. D'ora in avanti 
chiameremo tale dimensione $\omega$ e l'algoritmo derivante $\omega$-WFA. 
Scegliamo $\omega \geq \Delta/\delta$ e $\omega > k$. Dove $k$ è il numero di 
server a disposizione. 

Il nostro algoritmo andrà a calcolare la funzione lavoro esattamente nello 
stesso modo in cui viene calcolata nell'algoritmo \textsc{Work} con la 
differenza che alla richiesta $t$-esima:
\[
    work(t, A_t) = 
    \left\{ \begin{array}{l l}
        w_t(A_t) & \quad \text{se } t < \omega\\
        w_\omega(A_t) & \quad \text{se } t \geq \omega \\
    \end{array}\right.
\]
Considerate le premesse andremo a dimostrare il fatto che $\omega$-WFA è 
$(2k - 1)$-competitivo basandoci sull'assunzione che \textsc{Work} lo sia
perciò qualsiasi risultato migliore venga provato per \textsc{Work} si propaga
all'ottimizzazione in analisi.

\newtheorem{competitive}{Teorema}
\begin{competitive}
$\omega$-WFA è competitivo con lo stesso rapporto di competitività di 
\textsc{Work}.
\end{competitive}
\begin{proof}
Dividiamo la sequenza di richieste $R = \phi_0 \phi_1 ... \phi_n$ in fasi tali
che ogni fase inizia quando OPT\footnote{L'algoritmo ottimo offline} muove un 
server per rispondere ad una richiesta cioè paga almeno un costo $\delta$. 
Tranne che per per $\phi_n$ in cui ci sono al massimo $k$ richieste distinte.
Dividiamo l'analisi di una fase in due casi:
\begin{itemize}
    \item Se $|\phi_i| \leq \omega$ allora di sicuro $\omega$-WFA si comporta 
    come \textsc{Work} perché calcola tutta la funzione lavoro fino all'inizio 
    della fase. 
    \item Se $|\phi_i| > \omega$ allora ci sono al più $k$ richieste distinte 
    altrimenti avremmo che OPT muove un server. Essendo $\omega > k$ e 
    $\omega \delta > \Delta$ allora non ci sono posizioni irraggiungibili da 
    $\omega$-WFA che posizionerà i $k$ server nella stessa configurazione di
    OPT con lo stesso rapporto di competitività di \textsc{Work}.
\end{itemize}
\end{proof}

\end{document}
