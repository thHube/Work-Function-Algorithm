\documentclass[a4paper, 11pt]{article}

\usepackage[utf8]{inputenc}
\usepackage[italian]{babel}

\usepackage{amsmath}
\usepackage{amsthm}

\begin{document}

\title{Una semplice prova della $(2k - 1)$-competitività di $\omega$-WFA}
\maketitle

Vogliamo mostrare come l'ottimizzazione dell'algoritmo \textsc{Work} con una 
finestra di ricerca sia $(2k - 1)$-competitivo esattamente come l'algoritmo 
originale modulo alcune assunzioni. In primo luogo considereremo uno spazio
metrico finito, la cui distanza massima è $\Delta$ inoltre consideriamo una
discretizzazione di questo spazio tale che la distanza minima tra le richieste
si $\delta$. In generale queste assunzioni non sono limitanti dal momento che 
le implementazioni effettive sul calcolatore hanno intrinsecamente queste 
limitazioni.

La prima scelta che operiamo è la dimensione della finestra. D'ora in avanti 
chiameremo tale dimensione $\omega$ e l'algoritmo derivante $\omega$-WFA. 
Scegliamo $\omega \geq \Delta/\delta$ e $\omega > k$. Dove $k$ è il numero di 
server a disposizione. 

Il nostro algoritmo andrà a calcolare la funzione lavoro esattamente nello 
stesso modo in cui viene calcolata nell'algoritmo \textsc{Work} con la 
differenza che alla richiesta $t$-esima:
\[
    work(t, A_t) = 
    \left\{ \begin{array}{l l}
        w_t(A_t) & \quad \text{se } t < \omega\\
        w_\omega(A_t) & \quad \text{se } t \geq \omega \\
    \end{array}\right.
\]
Considerate le premesse andremo a dimostrare il fatto che $\omega$-WFA è 
$(2k - 1)$-competitivo basandoci sull'assunzione che \textsc{Work} lo sia
perciò qualsiasi risultato migliore venga provato per \textsc{Work} si propaga
all'ottimizzazione in analisi.

Andiamo ad effettuare una analisi al caso peggiore sfruttando un esempio fornito
da \cite{rudec}. Supponiamo la nostra configurazione sia la stessa. Abbiamo una 
``isola'' composta dalle posizioni ammissibili $x_1, x_2$ in cui arrivano le 
richieste con un server nella posizione $x_1$ (intercambiabilmente in $x_2$) 
e un ``entroterra'' in cui vi sono tutti gli altri server a distanza $\Delta$. 
e supponiamo di aver scelto $\omega = \Delta / \delta + 1$ allora alle prime 
$\omega - 1$ richieste $\omega$-WFA paqherà un costo pari a $\delta \omega$ e 
poi alla $w$-esima sposta un server dall'entroterra pagando un costo totale 
$(\omega - 1)\delta + \Delta$ mentre OPT off-line paga un costo $\Delta$ 
partendo dalla stessa configurazione. Perciò comparando i costi otteniamo che 
\[
    \frac{(\omega - 1) \delta + \Delta}{\Delta} \leq (2k -1) \Rightarrow 
    2  \leq (2k - 1)
\]
Dunque anche nel peggiore degli scenari l'algoritmo paga un costo inferiore 
a $(2k - 1)C_{OPT}$. Andiamo a dimostrare il caso generale, posta questa 
condizione, per induzione.

\newtheorem{competitivity}{Teorema}
\begin{competitivity}
$\omega$-WFA è $(2k - 1)$-competitivo.
\end{competitivity}
\begin{proof}
Andiamo ad effettuare la dimostrazione per induzione sulla lunghezza della 
richiesta riportandoci al fatto che \textsc{Work} è $(2k - 1)$-competitivo.
\begin{itemize}
    \item \emph{Caso base}. Per sequenze di richieste di lunghezza uno è banale
    dato che i due algoritmi (\textsc{Work} e $\omega$-WFA) hanno lo stesso 
    comportamento.
    \item \emph{Passo induttivo}. Abbiamo una sequenza di richieste di lunghezza
    $t > \omega$ (l'unico caso veramente interessate). Abbiamo dimostrato che 
    l'algoritmo anche al caso peggiore è $(2k - 1)$-competitivo dunque la 
    funzione lavoro verrà calcolata rispettando questo limite di competitività.
    Al passo $t - 1$ l'algoritmo calcola $w_\omega(A_{t-1})$ in modo competitivo
    allora, per ipotesi induttiva anche al passo $t$ è competitivo perché calcola
    $w_\omega(A_t) = w_\omega(A_{t-1}) + d(r_t, x)$ rispettando il vincolo 
    imposto dall'analisi al caso peggiore.
    
\end{itemize}
Questo conclude la dimostrazione.
\end{proof}

\bibliographystyle{alpha}
\bibliography{ref}

\end{document}
