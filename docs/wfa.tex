\documentclass[a4paper, 11pt]{article}

\usepackage[utf8]{inputenc}
\usepackage[italian]{babel}

\usepackage{amsmath}
\usepackage{amsthm}

\begin{document}

\title{Una semplice prova della $(2k - 1)$-competitività di $\omega$-WFA}
\maketitle

Vogliamo mostrare come l'ottimizzazione dell'algoritmo \textsc{Work} con una 
finestra di ricerca sia $(2k - 1)$-competitivo esattamente come l'algoritmo 
originale modulo alcune assunzioni. In primo luogo considereremo uno spazio
metrico finito, la cui distanza massima è $\Delta$ inoltre consideriamo una
discretizzazione di questo spazio tale che la distanza minima tra le richieste
si $\delta$. In generale queste assunzioni non sono limitanti dal momento che 
le implementazioni effettive sul calcolatore hanno intrinsecamente queste 
limitazioni.

La prima scelta che operiamo è la dimensione della finestra. D'ora in avanti 
chiameremo tale dimensione $\omega$ e l'algoritmo derivante $\omega$-WFA. 
Scegliamo $\omega \geq \Delta/\delta$ e $\omega > k$. Dove $k$ è il numero di 
server a disposizione. 

Il nostro algoritmo andrà a calcolare la funzione lavoro esattamente nello 
stesso modo in cui viene calcolata nell'algoritmo \textsc{Work} con la 
differenza che alla richiesta $t$-esima:
\[
    work(t, A_t) = 
    \left\{ \begin{array}{l l}
        w_t(A_t) & \quad \text{se } t < \omega\\
        w_\omega(A_t) & \quad \text{se } t \geq \omega \\
    \end{array}\right.
\]
Considerate le premesse andremo a dimostrare il fatto che $\omega$-WFA è 
$(2k - 1)$-competitivo basandoci sull'assunzione che \textsc{Work} lo sia
perciò qualsiasi risultato migliore venga provato per \textsc{Work} si propaga
all'ottimizzazione in analisi.

\newtheorem{reaching}{Lemma}
\begin{reaching}
Siano $\rho = r_1, ..., r_\omega$ richieste distinte. Allora
\[
    C_{\omega-\text{WFA}}(\rho) \leq C_{\text{OPT}}(\rho) + cost
\]
\end{reaching}
\begin{proof}
Ovvio dalla definizione di $\omega$-WFA.
\end{proof}
Questo lemma ci dice che, se dividiamo in fasi lunghe $\omega$ la sequenza di 
richieste (supponiamo siano $n$) alla fine sarà valida la disuquaglianza:
\[
    C_{\omega-\text{WFA}}(\rho) \leq C_{\text{OPT}}(\rho) + cost 
\]
la costante in questo caso sarà $cost \leq \Delta k n$. Dobbiamo dimostrare che 
la costante non aumenta linearmente con il numero di fasi. In questo caso 
l'algoritmo è competitivo come \textsc{Work}.

\end{document}
